\documentclass[11pt,a4paper,roman]{moderncv}
\moderncvstyle{classic}
\moderncvcolor{purple}

\usepackage[a4paper,top=2cm,bottom=2cm,outer=2cm,inner=2cm,verbose,heightrounded]{geometry}
\renewcommand{\link}[1]{{\texttt{\color{color1}\href{#1}{#1}}}}

\firstname{Federico}
\familyname{Dossena}

\title{Curriculum Vitæ}
\phone{(+39) 371 36 32 631}
\email{info@fdossena.com}
\homepage{fdossena.com}{https://fdossena.com}
\photo[70pt][0.3pt]{photo}

\begin{document}
\makecvtitle

\section{Studi}
\cventry{2018--2021}{Dottore Magistrale in Informatica}{Università degli Studi di Milano}{}{}{\textit{110 e lode / 110}}
\cventry{2010--2016}{Dottore in Informatica}{Università degli Studi di Milano}{}{}{\textit{110 e lode / 110}}
\cventry{2005--2010}{Diploma di Perito Informatico}{I.T.I.S. Galileo Galilei}{Crema}{}{\textit{95 / 100}}

\section{Tecnologie e capacità tecniche}
\cvitem{Avanzato}{Java, JavaScript, HTML, CSS, Git, LibreOffice, GNU/Linux, Microsoft Windows}
\cvitem{Intermedio}{C, PHP, Apache, MySQL, Android, Python, Docker, Bash, Arduino, Raspberry Pi \LaTeX, Markdown}
\cvitem{Base}{C++, Node.js, Nix, x86 Assembly, Reverse engineering (Ghidra, IDA Pro), Elettronica digitale e analogica}

\section{Esperienza professionale}
\cventry{2021--Presente}{Docente}{I.I.S. Galileo Galilei}{Crema}{}{Insegnante di discipline Informatiche a studenti di scuole medie superiori}
\cventry{2017--2021}{Sviluppatore freelance}{}{}{}{Sviluppo di siti web e applicazioni}
\cventry{2013--2018}{Vari ruoli}{MBstore distribuzione informatica}{Crema}{}{
    \begin{itemize}
    \item Tecnico informatico:\begin{itemize}
        \item Assemblaggio, riparazione e vendita di PC
        \item Assistenza presso privati e aziende
    \end{itemize}
    \item Sviluppatore:\begin{itemize}
        \item Sviluppo di siti e applicazioni web
        \item Sviluppo di strumenti software per uso interno
    \end{itemize}
    \end{itemize}
}
\cventry{Occasional-\\mente}{Ripetizioni private}{}{}{}{Ripetizioni per studenti di scuole medie superiori e occasionalmente universitari in materie informatiche in generale: programmazione, sistemi operativi, basi di dati, ecc.}

\section{Progetti}
Elenco completo disponibile su \link{https://fdossena.com} (in Inglese).
\newline{}

\cventry{2023--Presente}{Author, Developer}{TDF}{\link{https://github.com/adolfintel/tdf}}{}{Uno strumento di compatibilità per eseguire videogiochi per Microsoft Windows su sistemi GNU/Linux, simile a Proton di Valve}
\cventry{2021--Presente}{Autore, Sviluppatore}{OpenLDAT}{\link{https://openldat.fdossena.com}}{}{Strumento libero per la misurazione di metriche di latenza dei display LCD}
\cventry{2019--Presente}{Autore, Sviluppatore}{OpenPods}{\link{https://github.com/adolfintel/openpods}}{}{Applicazione Open Source per il monitoraggio delle Apple AirPods su dispositivi Android}
\cventry{2016--Presente}{Autore, Sviluppatore}{LibreSpeed}{\link{https://librespeed.org}}{}{Lo Speed Test Open Source più utilizzato al mondo}
\cventry{2015--Presente}{Autore, Sviluppatore, Blogger}{fdossena.com}{\link{https://fdossena.com}}{}{CMS ultraleggero in PHP per siti portfolio e blog}
\cventry{2014--Presente}{Autore, Maintainer}{WineD3D For Windows}{\link{https://wined3d.fdossena.com}}{}{Build di WineD3D per Microsoft Windows, uno strumento utile per la preservazione digitale di videogiochi}
\cventry{2014--2020}{Autore, Sviluppatore}{SINE Isochronic Entrainer}{\link{https://sine.fdossena.com}}{}{Sintetizzatore Open Source di toni isocronici per meditazione (sviluppo cessato)}

\section{Public speaking}
\cventry{2020--Presente}{Speaker}{Associazione HackLab Cormano}{\link{https://hacklabcormano.it}}{}{Presentazioni su vari argomenti inerenti al mondo informatico e elettronico}
\cventry{2019}{Speaker}{Workshop GARR 2019}{\link{https://fdossena.com/?p=wsgarr2019/i.md}}{}{Presentazione sul progetto LibreSpeed e la relativa collaborazione con GARR}
\cventry{2019}{Speaker}{ITNOG 5}{\link{https://fdossena.com/?p=itnog5/i.md}}{}{Presentazione sul progetto LibreSpeed}

\section{Pubblicazioni}
\cventry{2022}{OpenLDAT - A System for the Measurement of Display Latency Metrics}{Federico Dossena, Andrea Trentini}{JSID}{doi 10.1002/jsid.1104}{}

\section{Lingue parlate}
\cvitemwithcomment{Italiano}{Madrelingua}{}
\cvitemwithcomment{Inglese}{Avanzato}{Fluente}

\section{Altri interessi}
\renewcommand{\listitemsymbol}{-~}
\cvlistitem{Software Libero e Open Source}
\cvlistitem{Preservazione digitale}
\cvlistitem{Videogiochi}
\cvlistitem{Musica elettronica}
\cvlistitem{Stampa 3D}

\vspace*{\fill}
\rule[1ex]{\textwidth}{0.5pt}
Autorizzo il trattamento dei dati personali contenuti nel presente curriculum vitæ in base all'art. 13 del D.Lgs. 196/2003 e all'art. 13 GDPR 679/16.

\end{document}
